\section{Interface with compiled languages}
\label{sec:interface}

While we promote the advantages of Python as a numerical computing language, we also recognize the tremendous value and expertise already present in mature, battle-tested programs and libraries. In fact, the scientific Python community is the best example of how to combine both compiled and interpreted languages: since the very beginning, many Python libraries were written as wrappers to old FORTRAN or C++ code, using tools like f2py and SWIG (discussed below)\cite{Millman_2011}. It is therefore our intention to build on Python for new code and at the same time take advantage of available libraries written in different languages to avoid duplicating efforts.

\subsection{C and C++: ctypes, Cython, SWIG, CFFI}

\subsection{Fortran: f2py}

Although we could use an intermediate C wrapper to easily call Fortran code using the ISO C binding capabilities introduced in version 2003, there is no reliable way of doing this with Fortran 95 and earlier due to the lack of standardization. To solve this problem, specially for FORTRAN 77 code, the \verb|f2py| project was created in the early days of the scientific Python community, which wraps FORTRAN 77 and a subset of Fortran 95 directly in Python\cite{peterson2009f2py}.

\subsection{Others: Java, MATLAB}