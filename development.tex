\section{Development approach}
\label{sec:development}

\verb|poliastro| relies on well-tested, community-backed libraries for low level astronomical tasks, such as astropy\cite{Robitaille2013} and jplephem. In this section we comment the positive outcomes of the new open development strategies and the permissive, commercial-friendly licenses omnipresent in the scientific Python ecosystem.

\subsection{Free/Open Source software}

Software licensing is usually an underestimated topic that should not be taken lightly for the reasons explained below. In particular, some surveys suggest that a high percentage of the software available on the Internet has no license whatsoever\footnote{http://www.theregister.co.uk/2013/04/18/github\_licensing\_study/}, which, under modern copyright law, might mean the opposite of what original authors intended\footnote{https://opensource.com/law/13/8/github-poss-licensing}. This problem is more pervasive than anticipated: for example, the IERS software did not have a proper software license put in place until 2009\footnote{https://igscb.jpl.nasa.gov/pipermail/igsmail/2009/006005.html}, despite being extremely important in Flight Dynamics to compute changes of coordinate reference frames. Releasing works into the public domain is not a sensible choice, since copyright law is different from country to country and the implications remain unclear\cite{deazley2006rethinking}.

Open source software has a long history, with its roots dating back to the creation of the Free Sofware movement in the mid eighties\cite{Stallman:1985:GM}. A distinction is often made between "open source" and "free", in that the latter requires derivative works and linking programs to be released under the same license (which is often referred to as the "viral" nature of such licenses, see\cite{stallman2009viewpoint}). For obvious reasons, this distinction has profound implications on the commercial availability of the software.

Fortunately, the open source culture dominates in the scientific Python community, and is therefore safe to assume that most numerical Python libraries can be reused in commercial, closed source products\footnote{http://nipy.sourceforge.net/software/license/johns\_bsd\_pitch.html}. This is an important advantage, since it means that companies usually do not have to worry about licensing or linking issues regarding their products, as long as they ship the proper citations and disclaimers. It also poses a great challenge for open source maintainers, since commercial users do not always contribute code back and non-commercial users often feel entitled to ask for features to be developed for free. The question on how to make open source software sustainable is still open and research is ongoing, and requires a deep implication of the stakeholders.

\subsection{Open Development}

For some scientific software packages it is common practice to upload new releases to a website or FTP directory to make them available to users, while there is no public list of open issues and they have to be privately reported to a private email address. That is the case with the Standards of Fundamental Astronomy (SOFA) Software Collection.

However, many other software projects follow a more open approach, with various degrees of adoption and success, which we will refer to as "Open Development"\footnote{https://opendevelopmentmethod.org/}. Among the characteristics of this approach we can name:

\begin{itemize}
\item Carrying development discussions on public mailing lists,
\item Displaying a public list of open issues and known defects (as well as fixed ones),
\item Publishing the complete history of the project using source control management tools,
\item Performing public code reviews based on the previous two,
\item Using Continuous Integration environments and striving for a high rate of statement or branch coverage,
\item Embracing democratic and transparent decision making processes, with a focus on diversity and safety
\end{itemize}

The benefits of this approach to commercial development practices are being studied\footnote{http://paypal.github.io/InnerSourceCommons/}.
