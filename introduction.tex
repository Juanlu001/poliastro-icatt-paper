\section{Introduction}

The Python programming language has seen broad recognition in the last decade among scientists and academics, being one of the most popular languages in astronomy\cite{Robitaille2013} and for educational purposes\cite{guo2014python}. It is highly trusted in corporative environments as well as a tool for scripting, automating tasks and creating high level APIs and wrappers.

Despite its merits, Python is regarded as a "toy language" among teams where a strong background on compiled languages and scientific computing exist, due to its dynamical and interpreted nature. However, in many cases this is the result of preconceived ideas: most scientists and engineers that write numerical software (which often features a strong algorithmic component and has strong performance requirements) usually do not have any formal training on computer programming, let alone software engineering best practices\cite{Wilson2014}. In fact, in the Aerospace industry there is a sad track record of software failures (see for instance \cite{albee2000report} and \cite{lions1996report}) that could have been avoided by following better software and systems engineering practices.

When selecting a certain programming language for a specific problem, we as engineers have the obligation to consider as much information as possible and make an informed decision based on technical grounds. For example, if defect density were to be selected as the single figure to rank the contenders, well-established languages for space applications such as FORTRAN or C would perform worse than functional languages such as Haskell or Erlang\cite{Ray2014}. Another common misconception is to assume that each language features certain properties, while languages are abstract specifications and language \textit{implementations} are the concrete systems we can measure. Other metrics that could be taken into account are readability and programmer productivity, specially considering that "programmers write roughly the same number of lines of code per unit time regardless of the language they use"\cite{Wilson2014}.

In this paper we claim that the Python programming language, with the aid of both young projects and solid, well tested libraries, can be an optimal solution for the prototyping stage of the development and a fair complement to traditional alternatives in the production stage, in terms of performance, availability, maturity and maintainability. As a demonstrator, we selected a basic problem in Astrodynamics and compared the performance of existing FORTRAN or C++ implementations with our new Python implementations, comparing them in terms of code length and performance. In particular, we implemented two algorithms for solving the Lambert problem, as outlined in \cite{vallado2001fundamentals} and \cite{Izzo2014}, each already available free of charge on the Internet.

Our Python code is available as part of the latest version (v0.5) of the \verb|poliastro| package, an open source library dedicated to problems arising in Astrodynamics and Orbital Mechanics, such as orbit propagation, solution of the Lambert's problem, conversion between position and velocity vectors and classical orbital elements and orbit plotting, focused on interplanetary applications and released under the MIT license. The documentation and source code of the package are available online\footnote{https://github.com/poliastro/poliastro}\footnote{http://poliastro.readthedocs.org/en/v0.5.0/}.

This is a test cite \cite{Ziemer2012}. More info on sustainable scientific software development may be found on \cite{brown2015run}.
